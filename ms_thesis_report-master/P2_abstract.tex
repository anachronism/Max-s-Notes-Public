\begin{abstract}
	\par This work devises a novel approach for mitigating the effects of Catastrophic Forgetting in Deep Reinforcement Learning-based cognitive radio engine implementations employed in space communication applications. Previous implementations of cognitive radio space communication systems utilized a moving window-based online learning method, which discards part of its understanding of the environment each time the window is moved. This act of discarding is called Catastrophic Forgetting. This work investigated ways to control the forgetting process in a more systematic manner, both through a recursive training technique that implements forgetting in a more controlled manner and an ensemble learning technique where each member of the ensemble represents the engine’s understanding over a certain period of time. Both of these techniques were integrated into a cognitive radio engine proof-of-concept, and were delivered to the SDR platform on the International Space Station. The results were then compared to the results from the original proof-of-concept. Through comparison, the ensemble learning technique showed promise when comparing performance between training techniques at different EsNo values. 	

% 		Space-based communications systems to be employed by future articial satellites, or
% spacecraft during exploration missions, can potentially benet from software-dened radio
% adaptation capabilities. Multiple communication requirements could potentially compete
% for radio resources, whose availability of which may vary during the spacecraft's operational
% life span. Electronic components are prone to failure, and new instructions will eventually be
% received through software updates. Consequently, these changes may require a whole new set
% of near-optimal combination of parameters to be derived on-the-
% y without instantaneous
% human interaction or even without a human in-the-loop. Thus, achieving a suciently set of
% radio parameters can be challenging, especially when the communication channels change
% dynamically due to orbital dynamics as well as atmospheric and space weather{related
% impairments.
% This dissertation presents an analysis and discussion regarding novel algorithms propo-
% sed in order to enable a cognition control layer for adaptive communication systems ope-
% rating in space using an architecture that merges machine learning techniques employing
% wireless communication principles. The proposed cognitive engine proof-of-concept reasons
% over time through an ecient accumulated learning process. An implementation of the con-
% ceptual design is expected to be delivered to the SDR system located on the International
% Space Station as part of an experimental program.
% To support the proposed cognitive engine algorithm development, more realistic satellite-
% based communications channels are proposed along with rain attenuation synthesizers for
% LEO orbits, channel state detection algorithms, and multipath coecients function of the
% re
% ector's electrical characteristics. The achieved performance of the proposed solutions
% are compared with the state-of-the-art, and novel performance benchmarks are provided
% for future research to reference.
	\end{abstract}