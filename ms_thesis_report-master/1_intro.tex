\chapter{Introduction}
\section{Motivation}
\par In recent times, the amount of computing power available has increased to the point where many techniques that have previously been considered too complex for certain time-sensitive applications have become feasible \cite{computationPower_ML}. The increase in computing power includes the general purpose processors (GPPs) that most people are familiar with, but also graphics processing units (GPUs) and field-programmable gate arrays (FPGAs), which are more suited to specialized applications. One field with potential to benefit from these developments is space communications, for which an increase in computing power enables the use of Software Defined Radios (SDRs)\cite{tim1, tim2}. SDRs provide flexibility by enabling many aspects of communications that are typically implemented in hardware to be implemented in software. This includes tasks such as such as filtering and detection of signals, as well as higher level things like reconfiguration of applications and services. \cite[p.~xxxiii]{SDRTextbook}. The idea of software implementation of hardware elements is not particularly new, but was previously difficult to implement with the computing power available. As the computing power at any given price point increased, cognitive communication algorithms that utilize the manuverability of SDRs were proposed. Many of the algorithms that have been implemented and tested focus on employing dynamic spectrum access techniques or other sensing approaches. In more recent times, many technologies have taken a different approach by attempting to achieve multi-objective goals through the modification of multiple radio parameters in a manner that rewards good behaviors. This learning behavior sets a foundation for the development of cognitive systems. This increased focus on multi-objective goals has been in part motivated by the satellite communications industry, which has proposed business ventures involving hundreds of spacecraft. With satellite communications, objectives have the potential to interact in complicated ways and may be inversely related in some cases. Maintaining "good" performance requires a balancing of these objectives. These large networks of satellites, combined with the recent interest in space exploration, motivate communication methods that have high link availability and robustness\cite{paulo6}. A cognitive communications system is likely necessary in acheiving both of these requirements satisfatorily, especially for situations where autonomous operation is required.

\par One of the more general purpose technologies that has experienced large growth from the boom in processing power is Artificial Intelligence (AI), or more specifically Machine Learning (ML). Breakthroughs like IBM's Watson \cite{watsonPaper} and DeepMind's AlphaGo\cite{paulo5} have been driven by the ability to use large amounts of computing power to apply increasingly more complex Deep Learning (DL) techniques, such as convolutional neural networks (CNNs) and deep-Q networks (DQNs)\cite{paulo5}. These techniques have typically involved too many computations to be considered tractible for many problems. However, the recent explosion of computing power available has enabled the use of these DL techniques that reach performance and speed thresholds previously thought impossible. This allows for a whole variety of new fields of application that may have stricter timing or accuracy requirements. 


\section{State of the Art}
\par In this section, an overview of recent developments in satellite communications and cognitive radio as they relate to this thesis will be provided. 
\subsection{Satellite Communications}
\par Within the field of satellite communications, the most relevent developments are in the fields of Adaptive Coding and Modulation (ACM) schemes and satellite transmission standards.  ACM is used in the situation that received signal power changes due to impairments in the channel\cite{acm_explained}. This change in power is often a result of fading, which can be caused by weather conditions or the relative motion between the transmitter and the receiver \cite{paulo17}, among other things. ACM adaptively chooses the modulation and coding scheme based on the observed link budget and the quality of the message at the receiver, usually by looking for the optimal configuration in a table. ACM has been applied to 
Geosynchronous Equatorial Orbit (GEO) satellite channels operating in the S-band \cite{paulo18} as well as the Ka-band \cite{paulo19}. In this work, it is considered to be the standard solution for Low Earth Orbit (LEO) satellite links.  
%\textit{No direct comparisons to ACM will be made, as ACM is used primarily in situations where changes in the channel are somewhat predictable \textbf{[confirm this]}}.
\par In the context of satellite communications, there are a number of different standards that can be used in transmission of data. For this thesis, DVB-S2 \cite{paulo21} was chosen as the standard that represents most modern satellites. This standard is the second generation of DVB-S, a technical standard for GEO satellite-based digital television brodcast systems that was primarily desigend for direct-to-home services. Most of the innovations of DVB-S2 are present in the physical (PHY) layer in the OSI model \cite{OSIModel}, with more e-client channel coding, modulation, and error correction techniques. In addition, it uses recent video compression technology to enable transmissions compatable with MPEG-2 and MPEG-4 \cite{mp4_std} standards. DVB-S2 also utilizes a powerful forward error correction scheme that allows for four modulation constellations at a variety of code rates. The selection of modulation constellations and code rates can be controlled by an ACM scheme that allows for modulation on a frame-by-frame basis. In practice, DVB-S2 and its most recent extension DVB-S2X \cite{paulo22} have improved performance for mobile applications by allowing channel bonding, which combines unused portions of spectrum into a single virtual channel that can provide a higher bandwidth.
\subsection{Cognitive Radio}
\par Within the SDR literature, the concept of on-board cognition has been considered the likely next technological breakthrough.  On-board cognition enables environmental awareness across several Open Systems Interconnection (OSI) layers \cite{paulo39}, real-time knowledge of channel conditions, and assessment of currently available resources, among other things. This information is a very important aspect of optimizing the communications link performance. In the past,  a variety of different algorithms were applied to cognitive radios\cite{paulo41}, including genetic algorithms \cite{paulo40}. Genetic algorithms in particular do not always converge, and might take several iterations of the algorithm to find a stable solution. With each change in environmental conditions requiring another set of iterations to find the new stable solution, the time for training and retraining makes genetic algorithms unlikely to be a good fit for the rapidly updating environment of satellite communications, and CR in general.
\par ML techniques have been studied for use in CR \cite{paulo42,paulo45}. Both studies investigated how optimization and ML can be used in assisting CR systems to find the best configuration parameter set. The majority of research focuses on spectrum management and sensing techniques for terrestrial links \cite{paulo45,paulo47}. Research on satellite links has similar focus points, with the majority of CR research focusing on spectrum resource allocation \cite{paulo48,paulo50}. At the time of the writing of this thesis, there has been very little focus on radio resource managementfor point-to-point communication links.
\par For this thesis, communications performance is evaluated as a holistic combination of pre-existing performance metrics, including minimum Bit Error Rate(BER), maximum throughput, and power adaptation. During critical space mission phases, communications systems may need to manage resources while encountering conflicting performance requirements and limited resource availability. 
\section{Research Contributions}
\par In the research that directly precedes this thesis\cite{paulo_theory_paper,tim_implementation_paper}, a Cognitive Engine (CE) architecture for adapting PHY parameters for space communications was developed and tested. The CE combines a DQN with an "Explore" network modelling the relationship between the action space (which in this case is the configurable PHY parameters) and certain evaluation metrics relevant to communications. The DQN is used to choose a high quality action, while the Explore network is used to steer exploration of the environment to the subset of actions that it expects to get good results. By doing this, it is able to learn which actions are appropriate in any given channel condition.  
\par This thesis extends the previous research by investigating methods to mitigate Catastrophic Forgetting, a side effect of constantly retraining the DQN and the Explore network. Multiple different training methods, including a recursive method and an ensemble-based method, were developed and tested. First, a MATLAB simulation was used to verify the validity of recursive and ensemble methods as improvements on the base CE. Then, through C++ simulations and testing with the ISS, Learn++.NSE was determined to be the better way of addressing Catastrophic Forgetting, getting closer to the optimal fitness score than the baseline CE training method. 

\section{Thesis Outline}
\par In Chapter \ref{ch:bg}, relevant aspects of ML and the details about previous work on the project are described. Following this, details of the modified CE architecture, the MATLAB simulation, and the C++ implementation are described in Chapter \ref{ch:methods}, along with the hardware setups used for testing. Chapter \ref{ch:results} describes the simulation and flight testing results. Finally, Chapter \ref{ch:conclusion} summarizes the concluding points and describes future work.
